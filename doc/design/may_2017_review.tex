\documentclass{article}
\renewcommand{\c}[1]{\texttt{#1}}
\begin{document}

\section{The butler}

Adding audio to the preview means that we have an audio thread which
needs real-time access to audio data.  This means that we must have a
separate decoding thread.

This is implemented by having a \c{Butler} class which runs a
\c{Player}.  The \c{Butler} keeps ring buffers of video and
audio data filled by its own thread.  It tries to keep the ring
buffers full by calling \c{Player::pass()} when necessary.

The \c{Butler} fills its buffers and then waits for the
\c{Butler::\_summon} condition to be raised, at which point it
refills the buffers.

\section{Player}

\c{Player} has been returned to an API whereby the \c{Butler} calls
\c{pass} and then signals are emitted by \c{Player} to give data to \c{Butler}.

Users should call \c{Player::pass} which will call \c{Video},
\c{Audio} and \c{Subtitle} signals to be emitted.  These emissions
will happen in no particular order but should be fairly
closely-related in time (within \c{Butler::VIDEO\_READAHEAD} and
\c{Butler::AUDIO\_READAHEAD} of each other, I think).

Users may also call \c{Player::seek}.  If \c{accurate} is set to true
on this call, the following signal emissions will only be at or
after the seek time.  If \c{accurate} is false the following
emissions may be at any time after the seek time.

\c{Player} guarantees that emissions will represent continguous data.

\section{Decoders}

\c{Player} will request that \c{Decoder} classes seek as required.
After seeks \c{Decoder}s can emit whatever data they like, and
\c{Player} will discard them if they come before the time of an accurate
seek.

\section{General remarks}

The \c{Player} class should feature very heavily in unit testing as it
has significant influence.

\end{document}
