\documentclass{article}
\usepackage[usenames]{xcolor}
\usepackage{listings}
\title{Decoder structures}
\author{}
\date{}
\begin{document}
\maketitle

At the time of writing we have a get-stuff-at-this-time API which
hides a decode-some-and-see-what-comes-out approach.

\section{Easy and hard extraction of particular pieces of content}

With most decoders it is quick, easy and reliable to get a particular
piece of content from a particular timecode.  This applies to the DCP,
DCP subtitle, Image and Video MXF decoders.  With FFmpeg, however,
this is not easy.

This suggests that it would make more sense to keep the
decode-and-see-what-comes-out code within the FFmpeg decoder and not
use it anywhere else.

However resampling screws this up, as it means all audio requires
decode-and-see.  I don't think you can't resample in neat blocks as
there are fractional samples and other complications.  You can't postpone
resampling to the end of the player since different audio may be
coming in at different rates.

This suggests that decode-and-see is a better match, even if it feels
a bit ridiculous when most of the decoders have slightly clunky seek
and pass methods.

Having said that: the only other decoder which produces audio is now
the DCP one, and maybe that never needs to be resampled.


\section{Multiple streams}

Another thing unique to FFmpeg is multiple audio streams, possibly at
different sample rates.

There seem to be two approaches to handling this:

\begin{enumerate}
\item Every audio decoder has one or more `streams'.  The player loops
  content and streams within content, and the audio decoder resamples
  each stream individually.
\item Every audio decoder just returns audio data, and the FFmpeg
  decoder returns all its streams' data in one block.
\end{enumerate}

The second approach has the disadvantage that the FFmpeg decoder must
resample and merge its audio streams into one block.  This is in
addition to the resampling that must be done for the other decoders,
and the merging of all audio content inside the player.

These disadvantages suggest that the first approach is better.

One might think that the logical conclusion is to take streams all the
way back to the player and resample them there, but the resampling
must occur on the other side of the get-stuff-at-time API.


\section{Going back}

Thinking about this again in October 2016 it feels like the
get-stuff-at-this-time API is causing problems.  It especially seems
to be a bad fit for previewing audio.  The API is nice for callers,
but there is a lot of dancing around behind it to make it work, and it
seems that it is more `flexible' than necessary; all callers ever do
is seek or run.

Hence there is a temptation to go back to see-what-comes-out.

There are two operations: make DCP and preview.  Make DCP seems to be

\lstset{language=C++}
\lstset{basicstyle=\footnotesize\ttfamily,
        breaklines=true,
        keywordstyle=\color{blue}\ttfamily,
        stringstyle=\color{red}\ttfamily,
        commentstyle=\color{olive}\ttfamily}

\begin{lstlisting}
  while (!done) {
    done = player->pass();
    // pass() causes things to appear which are
    // sent to encoders / disk
  }
\end{lstlisting}

And preview seems to be

\begin{lstlisting}
  // Thread 1
  while (!done) {
    done = player->pass();
    // pass() causes things to appear which are buffered
    sleep_until_buffers_empty();
  }

  // Thread 2
  while (true) {
    get_video_and_audio_from_buffers();
    push_to_output();
    sleep();
  }
\end{lstlisting}

\texttt{Player::pass} must call \texttt{pass()} on its decoders.  They
will emit stuff which \texttt{Player} must adjust (mixing sound etc.).
Player then emits the `final cut', which must have properties like no
gaps in video/audio.

Maybe you could have a parent class for simpler get-stuff-at-this-time
decoders to give them \texttt{pass()} / \texttt{seek()}.

One problem I remember is which decoder to \texttt{pass()} at any given time:
it must be the one with the earliest last output, presumably.
Resampling also looks fiddly in the v1 code.


\section{Having a go}

\begin{lstlisting}
  class Decoder {
    virtual void pass() = 0;
    virtual void seek(ContentTime time, bool accurate) = 0;

    signals2<void (ContentVideo)> Video;
    signals2<void (ContentAudio, AudioStreamPtr)> Audio;
    signals2<void (ContentTextSubtitle)> TextSubtitle;
  };
\end{lstlisting}

or perhaps

\begin{lstlisting}
  class Decoder {
    virtual void pass() = 0;
    virtual void seek(ContentTime time, bool accurate) = 0;

    shared_ptr<VideoDecoder> video;
    shared_ptr<AudioDecoder> audio;
    shared_ptr<SubtitleDecoder> subtitle;
  };

  class VideoDecoder {
    signals2<void (ContentVideo)> Data;
  };
\end{lstlisting}

Questions:
\begin{itemize}
\item Video / audio frame or \texttt{ContentTime}?
\item Can all the subtitle period notation code go?
\end{itemize}

\subsection{Steps}

\begin{itemize}
  \item Remove \texttt{get()}-based loops and replace with \texttt{pass()} and signal connections.
  \item Remove \texttt{get()} and \texttt{seek()} from decoder parts; add emission signals.
  \item Put \texttt{AudioMerger} back.
  \item Remove \texttt{during} stuff from \texttt{SubtitleDecoder} and decoder classes that use it.
  \item Rename \texttt{give} methods to \texttt{emit}.
  \item Remove \text{get} methods from \texttt{Player}; replace with \texttt{pass()} and \texttt{seek()}.
\end{itemize}

\end{document}
